\documentclass[a4paper,12pt]{article}
\usepackage[utf8]{inputenc}
\usepackage[spanish]{babel}
\usepackage{microtype} 

\usepackage{graphicx}
\usepackage{wrapfig}

\usepackage{amssymb}
\usepackage{amsmath} %For being able to make comments inside the formulas as normal text
\usepackage{amsthm}

\usepackage{cancel} %in the preamble gives you four different modes of striking through

\usepackage[a4paper, inner= 2cm, outer= 2cm,
top= 2cm, bottom= 2cm]{geometry}
\usepackage{fancyhdr}
\usepackage{animate}

\title{Taller 1\\ Probabilidad y Estadística}
\author{David Alsina, Juan José Caballero, Nicolas Botero}
\date{Enero 2021}

\usepackage[dvipsnames]{xcolor}
\definecolor{blueMacc}{RGB}{61, 160, 250}

\begin{document}

    \begin{figure}[ht]
		\centering
		\includegraphics[width = 17cm]{../header.png}
		\maketitle
    \end{figure}

	\textbf{5.} Utilice los axiomas de probabilidad para demostrar que, para dos eventos A y B:

	\begin{equation*}
		P((A \cap  \text{\={B}}) \cup (\text{\={A}} \cap B)) = P(A) + P(B) - 2P(A\cap B)
	\end{equation*}

	que es la probabilidad de que exactamente uno de los dos eventos A o B ocurra.

	\begin{align*}
		P((A \cap  \text{\={B}}) \cup (\text{\={A}} \cap B)) &= P( (A \cup \text{\={A}}) \cap (\text{\={A}} \cup \text{\={B}}) \cap (B \cup A) \cap (B \cup \text{\={B}}) )\\
		&= P( (\Omega) \cap (\text{\={A}} \cup \text{\={B}}) \cap (B \cup A) \cap (\Omega) )\\
		&= P( (A \cap B)^{c} \cap (A \cup B))\\
 	\end{align*}

	\begin{center}
		Ahora observe la siguiente ecuación:
	\end{center}
	\begin{align}
		P(A \cup  B) &= P(A) +  P(B) - P(A \cap B)\\
		P(A \cap B) &=  P(A) +  P(B) - P(A \cup B)
	\end{align}

	\begin{center}
		Dada la ecuación anterior (2) podríamos calcular esta probabilidad así:
	\end{center}

	\begin{align*}
		P( (A \cap B)^{c} \cap (A \cup B)) &= P((A \cap B)^{c}) + P(A \cup B) 
	- P((A \cap B)^{c} \cup (A \cup B)) \\
	&= 1 - P(A \cap B) + P(A) + P(B) - P(A \cap B)- P((A \cap B)^{c} \cup (A \cup B))\\
	&= P(A) + P(B) -2P(A \cap B) + 1 - P((A \cap B)^{c} \cup (A \cup B))\\
	&= P(A) + P(B) -2P(A \cap B) + 1 - P(\Omega)\\
	&= P(A) + P(B) -2P(A \cap B) + 1 - 1\\
	&= P(A) + P(B) -2P(A \cap B)_{\;\;\blacksquare}
	\end{align*}

	\newpage		%salto de página

	\textbf{4.}	Comencemos por mirar todos los casos de este problema:\\

	\textbf{Caso 1:} El más fácil va de primero.
	En este caso la única opción viable para poder ganar la final sería ganarle al segundo.\\

	\textbf{Caso 2:} El más fácil va de tercero.
	En este caso la única opción viable para poder ganar la final es ganándole al segundo.\\

	\textbf{Caso 3}: El más fácil de segundo.
	Este es el caso más viable ya que tiene 2 opciones para ganar la final que sería ganarle al primero o ganarle al tercero. En este caso la unión de estás 2 probabilidades es mayor a la de cada una por singular.\\

	Entonces por esto último, la forma más viable de ganar la final es dejar al más fácil de segundo.



\end{document}

