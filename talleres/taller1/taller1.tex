\documentclass[a4paper,12pt]{article}
\usepackage[utf8]{inputenc}
\usepackage[spanish]{babel}
\usepackage{microtype} 

\usepackage{graphicx}
\usepackage{wrapfig}

\usepackage{amssymb}
\usepackage{amsmath} %For being able to make comments inside the formulas as normal text
\usepackage{amsthm}

\usepackage{cancel} %in the preamble gives you four different modes of striking through

\usepackage[a4paper, inner= 2cm, outer= 2cm,
top= 2cm, bottom= 2cm]{geometry}
\usepackage{fancyhdr}
\usepackage{animate}

\title{Taller 1\\ Probabilidad y Estadística}
\author{David Alsina, Juan José Caballero, Nicolas Botero}
\date{Enero 2021}

\usepackage[dvipsnames]{xcolor}
\definecolor{blueMacc}{RGB}{61, 160, 250}

\begin{document}

    \begin{figure}[ht]
		\centering
		\includegraphics[width = 17cm]{../header.png}
		\maketitle
    \end{figure}

	\textbf{5.} Utilice los axiomas de probabilidad para demostrar que, para dos eventos A y B:

	\begin{equation*}
		P((A \cap  \text{\={B}}) \cup (\text{\={A}} \cap B)) = P(A) + P(B) - 2P(A\cap B)
	\end{equation*}

	que es la probabilidad de que exactamente uno de los dos eventos A o B ocurra.

	\begin{align*}
		P((A \cap  \text{\={B}}) \cup (\text{\={A}} \cap B)) &= P( (A \cup \text{\={A}}) \cap (\text{\={A}} \cup \text{\={B}}) \cap (B \cup A) \cap (B \cup \text{\={B}}) )\\
		&= P( (\Omega) \cap (\text{\={A}} \cup \text{\={B}}) \cap (B \cup A) \cap (\Omega) )\\
		&= P( (A \cap B)^{c} \cap (A \cup B))\\
 	\end{align*}

	\begin{center}
		Ahora observe la siguiente ecuación:
	\end{center}
	\begin{align}
		P(A \cup  B) &= P(A) +  P(B) - P(A \cap B)\\
		P(A \cap B) &=  P(A) +  P(B) - P(A \cup B)
	\end{align}

	\begin{center}
		Dada la ecuación anterior (2) podríamos calcular esta probabilidad así:
	\end{center}

	\begin{align*}
		P( (A \cap B)^{c} \cap (A \cup B)) &= P((A \cap B)^{c}) + P(A \cup B) 
	- P((A \cap B)^{c} \cup (A \cup B)) \\
	&= 1 - P(A \cap B) + P(A) + P(B) - P(A \cap B)- P((A \cap B)^{c} \cup (A \cup B))\\
	&= P(A) + P(B) -2P(A \cap B) + 1 - P((A \cap B)^{c} \cup (A \cup B))\\
	&= P(A) + P(B) -2P(A \cap B) + 1 - P(\Omega)\\
	&= P(A) + P(B) -2P(A \cap B) + 1 - 1\\
	&= P(A) + P(B) -2P(A \cap B)_{\;\;\blacksquare}
	\end{align*}

	\newpage		%salto de página

	\textbf{4.}	Comencemos por modelar la situación: Inicialmente sabemos que se tienen 3 oponentes, sean estos
	A, B y C, de los cuales B es el más débil.\\

	Ahora nombremos a los eventos $A, B, C$ como la probabilidad de ganar a los oponentes de correspondiente letra 

	A priori se sabe que estos eventos son independientes entre sí es decir que:

	\begin{equation*}
		P(A\cap B) = P(A) \cdot P(B) 
	\end{equation*}

	Adicionalmente se sabe que $P(B) > P(A)$ y  $P(B) > P(C)$ esto dado que B es el más débil.
	Ahora haciendo un análisis por caso se tiene:


	\begin{enumerate}
		\item Se selecciona jugar contra el más débil como primer oponente, así la probabilidad de ganar G es:

			\begin{align*}
				P(G) &= P(B)\cdot P(A) + P(A)\cdot P(C)\\
				&= P(A)(P(B) + P(C))
			\end{align*}

		\item Se selecciona jugar contra el más débil como segundo oponente, así la probabilidad de ganar G es:
			\begin{align*}
				P(G) &= P(A)\cdot P(B) + P(B)\cdot P(C)\\
				&= P(B)(P(A) + P(C))
			\end{align*}

		\item Se selecciona jugar contra el más débil como tercer oponente, así la probabilidad de ganar G es:
			\begin{align*}
				P(G) &= P(C)\cdot P(A) + P(A)\cdot P(B)\\
				&= P(C)(P(A) + P(B))
			\end{align*}
	\end{enumerate}

	(Note como se ha descartado el caso trivial en el que se vencen a todos los oponentes)
	Adicionalmente basado en que $P(B) > P(A)$ y  $P(B) > P(C)$:

	Comparando sin perdida de generalidad el primer caso con el segundo se tiene:

	\begin{align*}
			P(A)\cdot P(B) + P(B)\cdot P(C)_{\text{    2do caso}}	&> P(B)\cdot P(A) + P(A)\cdot P(C)_{\text{  1er caso}}\\
			P(B)\cdot P(C) &> P(A)\cdot P(C)
	\end{align*}

	dado lo anterior se concluye que el caso en el que se maximizan las posibilidades de ganar es poniendo 
	al jugador más débil como segundo oponente.


\end{document}

